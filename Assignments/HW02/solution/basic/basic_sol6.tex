% page of 98 book for canonical difinition

برای حل این سوال می‌توان از دو روش:
\begin{enumerate}
	\item 
	جبر بول
	\item 
	 جدول کارنو
\end{enumerate}
استفاده کرد که هردوی آن صحیح است. ما در این سوال از جدول کارنو استفاده می‌کنیم. تنها کافیست مینترم/ماکسترم متناظر با تابع داده را پیدا کنیم:



\begin{latin}
	\begin{minipage}{0.48\textwidth}
		\centering
		\begin{karnaugh-map}[4][2][1][$B$][$A$][$C$](label=corner)
			\minterms{1,2,3,7}
			\implicant{1}{3}
			\implicant{3}{2}
			\implicant{3}{7}
		\end{karnaugh-map}
		\caption{K-Map 1}
		$f(A,B,C)=AB+AC'+BC'$ or\\
		$f(A,B,C)=\sum m(2,4,6,7)$ or\\
		$f(A,B,C)=(A+B) \cdot (B+C') \cdot C'$ or\\
		$f(A,B,C)=\prod M(0,4,5,6)$
	\end{minipage}
	\hfill
	\begin{minipage}{0.48\textwidth}
		\centering
		\begin{karnaugh-map}[4][2][1][$B$][$A$][$C$](label=corner)
			\maxterms{0,2,3,4,6,7}
			\implicant{3}{6}
			\implicantedge{0}{4}{2}{6}
		\end{karnaugh-map}
		\caption{K-Map 1}
		$f(A,B,C)=A'B$ or\\
		$f(A,B,C)=\sum m(1,5)$ or\\
		$f(A,B,C)=(A)\cdot (B)$ or\\
		$f(A,B,C)=\prod M(0,2,3,4,6,7)$
	\end{minipage}	
\end{latin}

\begin{latin}
	\centering
	\begin{karnaugh-map}[4][4][2][$C$][$B$][$E$][$D$][$A$](label=corner)
		\minterms{4,5,20,21,28,29,24,25,23,22,30}
		\implicant{4}{21}
		\implicant{28}{25}
		\implicant{20}{22}
		\implicantedge{20}{28}{22}{30}
	\end{karnaugh-map}
	\caption{K-Map 1}
	$f(A,B,C)=AD'E+ADB'+AEC'+D'EB'$ or\\
	$f(A,B,C)=\sum m(1,5,17,18,19,21,22,23,25,27,29)$ or\\
	$f(A,B,C)=(A'+D+E') \cdot (A'+D'+B) \cdot (A'+E'+C) \cdot (D+E'+B) $ or\\
	$f(A,B,C)=\prod M(0,2,3,4,6,7,8,9,10,11,12,13,14,15,16,20,24,26,28,30)$
\end{latin}